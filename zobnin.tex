\documentclass[a4paper,12pt,russian]{amsart}
\usepackage[utf8]{inputenc}
\usepackage[T2A]{fontenc}
\usepackage[russian]{babel}
\usepackage{enumitem}
\usepackage{amssymb}

\righthyphenmin=2
\sloppy

\usepackage{geometry}
\geometry{a4paper,top=1.5cm,bottom=1cm,left=1.5cm,right=1.5cm}

\DeclareMathOperator{\ord}{ord}

\newcommand{\x}[3]{x_{#1, #3}^{#2}}

\renewcommand{\ge}{\geqslant}
\renewcommand{\le}{\leqslant}

\theoremstyle{definition}
\newtheorem{theorem}{Теорема}
\newtheorem{proposition}{Предложение}

\begin{document}

Пусть $f \in \mathbb{Q}\{y\}$~--- дифференциальный многочлен порядка $l$.
Мы будем пользоваться следующими фактами:

\begin{proposition}
При $n > 2p$
$$
 f^{(n)} = S_{f,n,0} y_{l+n} + S_{f,n,1} y_{l+n-1} + \ldots + S_{f,n,p} y_{l+n-p} + T_n,
$$
где
$$
 S_{f,n,k} = \sum_{j=\max(l-k,0)}^n C_{n}^{k-l+j} \left(\frac{\partial f}{\partial y_{j}}\right)^{(k-l+j)}
$$
и $\ord T_n < l+n-p$. $\qed$
\end{proposition}

\begin{proposition}
Если $S$~--- идеал, а $I$~--- радикальный идеал, то $I = I:S \cap (I+S)$. $\qed$
\end{proposition}

\begin{proposition}
Если $I$~--- идеал, то $I : (g_1, \ldots, g_n) = \bigcap_{i=1}^n I:(g_i)$. $\qed$
\end{proposition}

\begin{proposition}
\label{Primes}
Если $Q_i$~--- идеалы, $P$~--- простой идеал и $\bigcap_{i=1}^n Q_i \subset P$, то для некоторого $j$ $Q_j \subset P$. $\qed$
\end{proposition}

В этих предложениях идеалы рассматриваются в произвольном коммутативном кольце.

\bigskip

Пусть $h_1, \ldots, h_t$~--- произвольные многочлены, образующие идеал $J$.
Пусть $G_k$~--- идеал, порожденный элементами $\left(\frac{\partial f}{\partial y_j}\right)^{(k-l+j)}$, где $\max(l-k, 0) \le j \le l$.
Пусть также $S_p = G_0 + \ldots + G_p$.
Предположим, что $[f] + S_p = (1)$.
Требуется выяснить, равен ли идеал $I = [f] + J$ единичному.

\bigskip



\bigskip


Пусть $I_k = (f_0, \ldots, f_k) + J$. Заметим, что 
$$
 I = (1) \iff \exists \, k: I_k = (1) \iff \exists \, k: \sqrt{I_k} = (1).
$$
Выберем $k$ таким, чтобы $(f_0, \ldots, f_k) + S_p = (1)$, и чтобы $k$ превосходило порядки всех многочленов $h_i$ и число $2p$.

Разложим идеал $\sqrt{I_k}$:
$$
 \sqrt{I_k} = \sqrt{I_k} : S_p  \cap  \left(\sqrt{I_k} + S_p\right).
$$
Так как $\sqrt{I_k} + S_p = (1)$, то
$$
 \sqrt{I_k} = \sqrt{I_k} : S_p = \sqrt{I_k} : (G_0 + \ldots + G_d) = \bigcap\limits_{d = 0}^p \sqrt{I_k} : G_d.
$$
Чтобы доказать, что этот идеал отличен от $(1)$, необходимо и достаточно доказать, что один из идеалов $\sqrt{I_k}:G_d$ отличен от $(1)$.
Будем перебирать идеалы $\sqrt{I_k} : G_d$, увеличивая $d$.

Если $\sqrt{I_k} : G_d = (1)$, то и $\sqrt{I_n} : G_d = (1)$ для всех остальных $n \ge k$.
Будем считать, что $d$~--- наименьший номер, такой, что $\sqrt{I_k} : G_d \ne (1)$.
Наша цель~--- постараться получить аналогичное неравенство для $k+1$.

Идеал $\sqrt{I_k} : G_d$ является радикальным. Разложим его в минимальное пересечение простых идеалов $P_{k,d,i}$.
Это в точности минимальные простые компоненты идеала $\sqrt{I_k}$, не содержащие $G_d$.
Предположим, что среди этих простых идеалов существует идеал $P$, не содержащий ни одной обобщенной сепаранты $S_{f,n,d}$.
Покажем, что тогда идеал $\sqrt{I_{k+1}}:G_d$ также является собственным, и среди его минимальных простых также есть идеал $Q$,
не содержащей обобщенных сепарант. В таком случае по индукции мы получим, что $I \ne 1$.

Действительно, пусть $z = y_{l+k+1-d}$~--- старшая переменная многочлена $f_{k+1}$ по модулю $G_0, \ldots, G_{d-1}$
и $R$~--- подкольцо кольца многочленов, порожденное переменными, младшими $z$. 
Образующие идеалов $I_k, G_d$ и $P$ лежат в $R$.
Тогда $f_{k+1} \equiv S_{f,k+1,d} z + T \pmod P$, где $T \in R$.
Если $h \in I_{k+1} \cap R$, то верно равенство
$h \equiv q f_{k+1} \pmod P$, причем $q S_{f,k+1,d} \in P$.
Так как по предположению $S_{f,k+1,d} \notin P$ и $P$ является простым, то $q \in P$,
откуда $h \in P$. Итак,
\begin{gather*}
 I_{k+1} \cap R \subset P \; \Longrightarrow \; \sqrt{I_{k+1}} \cap R \subset \sqrt{P} = P \; \Longrightarrow \\
 \Longrightarrow \; \sqrt{I_{k+1}}:G_d \cap R \subset P:G_d = P \; \Longrightarrow \; \sqrt{I_k}:G_d \subset \sqrt{I_{k+1}}:G_d \cap R \subset P.
\end{gather*}
Пусть $\sqrt{I_{k+1}}:G_d = \cap_j Q_j$~--- минимальное разложение на простые компоненты.
Тогда все идеалы $Q_j \cap R$ тоже простые, причем
$$
 \sqrt{I_k}:G_d \subset \bigcap_j \left( Q_j \cap R \right) \subset P.
$$
Отсюда следует, что найдется $Q$, такой, что $Q \cap R = P$, так как $P$~--- минимальный простой.
Так как все обобщенные сепаранты лежат в $R$ и не лежат в $P$, то они не лежат и в $Q$.

\bigskip

Предположим теперь, что для каждого минимального простого идеала $P_{k,d,i}$
найдется принадлежащая ему обобщенная сепаранта $S_{f,n,d}$, где $n \ge k$.
Заметим, что все обобщенные сепаранты лежат в идеале $G_d$.
Выражение для этих обобщенных сепарант является многочленом по $n$ степени $d$,
не равным тождественно нулю по модулю каждого идеала $P_{k,d,i}$ (иначе получилось бы противоречие $G_d \subset P_{k,d,i}$).
Отсюда следует, что лишь для конечного набора значений $n \ge k$ обобщенная сепаранта $S_{f,n,d}$ может оказаться в идеале $P_{k,d,i}$.
Пусть число $K$ больше всех этих значений. Перейдем от $k$ к $K$ и повторим процедуру. Покажем, что такой повтор может произойти не более $d+1$ раз. Действительно, пусть имеется цепочка наборов минимальных простых идеалов $P_{k_j, d, i}$ для идеалов $\sqrt{I_{k_j}}:G_d$ длины $d+1$
$$
 \bigcap_i P_{k_1, d, i} \subset \bigcap_i P_{k_2, d, i} \subset \ldots \subset \bigcap_i P_{k_{d+1}, d, i},
$$
причем в каждом из них содержится некоторая обобщенная сепаранта.
Выберем по предложению~\ref{Primes} цепочку вложенных простых идеалов:
$$
 P_{k_1, d, i_1} \subset P_{k_2, d, i_2} \subset \ldots \subset P_{k_{d+1}, d, i_{d+1}}.
$$
Рассмотрим выражение для обобщенной сепаранты по модулю последнего идеала в этой цепочке. Это не тождественный ноль,
а некоторый многочлен по $n$ степени не выше $d$, имеющий корень по крайней мере в $d+1$ точке, что немедленно приводит нас к противоречию.
Итак, по крайней мере за $d$ таких повторов мы либо убедимся, что $\sqrt{I_k}:G_d = (1)$, либо получим простую компоненту этого идеала, не содержащую никаких обобщенных сепарант, и тогда по доказанному выше $I \ne (1)$.

\end{document}
