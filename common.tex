\documentclass[11pt]{article}

\usepackage[dvipsnames]{xcolor}
\usepackage[T2A]{fontenc}
\usepackage{amsfonts, amsmath, amsthm, amssymb}
\usepackage[utf8]{inputenc}
\usepackage[russian]{babel}
\usepackage[noend]{algpseudocode}
\usepackage{algorithm}

\usepackage{geometry}
\geometry{a4paper,top=2cm,bottom=2cm,left=2cm,right=2cm}

\DeclareMathOperator{\ord}{ord}

\renewcommand{\le}{\leqslant}
\renewcommand{\ge}{\geqslant}

% Стиль для теорем. Заголовок жирным, текст курсивом.
\theoremstyle{plain1}
\newtheorem{theorem1}{Теорема}
\newtheorem{lemma}[theorem1]{Лемма}
\theoremstyle{plain2}
\newtheorem{theorem2}{Теорема}
\theoremstyle{plain}
\newtheorem{theorem}{Теорема}
\theoremstyle{plain3}
\newtheorem{theorem3}{Теорема}
\newtheorem{proposition}[theorem1]{Предложение}
\newtheorem{corollary}[theorem1]{Следствие}
\newtheorem*{theorem*}{Теорема}
\newtheorem*{lemma*}{Лемма}
\newtheorem*{proposition*}{Предложение}
\newtheorem*{corollary*}{Следствие}

% Стиль для определений. Заголовок жирным, текст обычным.
\theoremstyle{definition}
\newtheorem{definition}[theorem2]{Определение}
\newtheorem{example}[theorem3]{Пример}
\newtheorem*{definition*}{Определение}
\newtheorem*{example*}{Пример}

% Стиль для замечаний. Заголовок курсивом, текст обычным.
\theoremstyle{remark}
\newtheorem{remark}[theorem1]{Замечание}

\newtheorem*{remark*}{Замечание}


\title{Алгоритм проверки тривиальности <<смешанных>> идеалов\\ в кольце дифференциальных многочленов}
\author{А. И. Зобнин \and М. А. Лимонов}

\begin{document}
\maketitle

\begin{abstract}
В статье предлагается алгоритм проверки тривиальности идеала
$[f] + (h_1, \ldots, h_t)$ в обыкновенном кольце дифференциальных многочленов
при некотором дополнительном условии на многочлен $f$.
Эта задача тесно связана, с одной стороны, с задачей Колчина об экспонентах дифференциальных идеалов,
а с другой~--- с вопросами конечности дифференциальных стандартных базисов.
\end{abstract}

\section{Введение}

В одной из своих ранних работ~\cite{Kolchin41} Эллис Колчин изучал возможные значения экспонент дифференциальных идеалов особого вида.
Экспонентой идеала $I$ называется минимальное натуральное число $m$, такое, что $\sqrt{I}^m\subset I$ (если такого числа не существует,
то экспонента считается равной бесконечности).
Колчин рассматривал идеалы, порождённые дифференциальным многочленом первого порядка.
В его работе было разобрано несколько достаточно общих случаев (но не все), и в этих случаях было установлено, что экспонента может равняться 1, 2 или~$\infty$.
В частности им рассматривались дифференциальные многочлены, не имеющие сингулярных корней, то есть такие, что
\begin{gather}
\label{nonsingular}
[f,S_f]\ne(1),
\end{gather}
где $S_f$~--- сепаранта~$f$ (частная производная многочлена $f$ по старшей переменной).
Колчин установил, что если, вдобавок, многочлен $f$ удовлетворяет условию
\begin{gather}
\label{strong}
[f]+(S_f)=(1),
\end{gather}
то экспонента идеала $[f]$ равна 1, то есть, идеал $[f]$ является радикальным (этот результат справедлив для многочленов $f$ любого порядка).
В то же время случай
\begin{gather}
\label{nontrivial}
[f, S_f] = (1), \quad [f] + (S_f) \ne (1)
\end{gather}
Колчин оставил неразобранным. Он даже не предложил никакого практического способа проверки того,
выполняется ли условие~\eqref{strong} при предположении~\eqref{nonsingular}.

С другой стороны, в 2007 году Дмитрий Трушин, изучая так называемый идеал сепарант, доказал~\cite{Trushin},
что при предположении~\eqref{nonsingular} условие~\eqref{strong} равносильно наличию в идеале квазилинейного многочлена
(т.~е. дифференциального многочлена, разрешённого относительно старшей производной).
В свою очередь, это равносильно существованию у идеала $[f]$ конечного дифференциального стандартного базиса (аналога базиса Грёбнера)
при порядках, близких в некотором смысле к лексикографическому~\cite{Zobnin}.

Как мы видим, в этих задачах возникает одно и то же условие~\eqref{strong} на многочлен $f$.
Если предположение~\eqref{nonsingular} можно проверить алгоритмически с помощью алгоритма Розенфельда--Грёбнера~\cite{BLOP, Sit},
то готового алгоритма для проверки тривиальности <<смешанной>> суммы дифференциального и недифференциального идеалов, т.~е. проверки условия~\eqref{strong}, не было известно.
Более того, высказывались предположения\footnote{Частная беседа одного из авторов с участниками конференции Gr\"obner Bases in Symbolic Analysis, Линц, Австрия, 2006.}, что такая задача алгоритмически неразрешима.

В данной работе мы предъявим алгоритм, проверяющий условие~\eqref{strong} (а на самом деле, решающий более общую задачу):
при предположении
\begin{gather}
\left[f, \frac{\partial f}{\partial y_0}, \ldots, \frac{\partial f}{\partial y_l} \right] = (1),
\end{gather}
где $l$~--- порядок дифференциального многочлена $f$,
проверить, верно ли равенство
\begin{gather}
\label{strong_general}
[f] + (h_1, \ldots, h_t) = (1),
\end{gather}
где $h_1, \ldots, h_t$~--- произвольные заданные многочлены.
Идеалы вида $[f] + (h_1, \ldots, h_t)$ мы будем называть <<смешанными>>: вообще говоря, они не являются дифференциальными.

Статья организована следующим образом. В главе~\ref{preliminaries} мы приводим необходимые определения из дифференциальной алгебры.
В главе~\ref{generalized_separants} строится разложение производной дифференциального многочлена по старшим переменным
с помощью так называемых обобщенных сепарант. В их терминах в главе~\ref{algorithm} формулируется алгоритм,
решающий поставленную задачу. В главе~\ref{examples} приводится несколько примеров, иллюстрирующих работу алгоритма.



\section{Основные определения и обозначения}
\label{preliminaries}
Мы будем работать над алгебраически замкнутым полем констант ${K}$ нулевой характеристики.

\begin{definition}
Кольцо $\mathcal{R}$ называется \emph{алгеброй Ритта}, если оно является алгеброй над~$\mathbb{Q}$.
\end{definition}

\begin{definition}
Отображение $\,\delta: \mathcal{R} \to \mathcal{R}$ называется \emph{дифференцированием}, если для любых элементов $a$ и $b$ кольца $\mathcal{R}$
выполнены следующие свойства:
\begin{enumerate}
  \item $\,\delta(a+b)= \,\delta(a)+\,\delta(b)$~--– линейность;
  \item $\,\delta(ab)=a\,\delta(b)+b\,\delta(a)$~--- правило Лейбница.
\end{enumerate}
\end{definition}

\begin{definition}
Кольцо $\mathcal{R}$ называется \emph{обыкновенным дифференциальным кольцом}, если на нем задано одно дифференцирование.
\end{definition}

\begin{definition}
Пусть $K$~--- обыкновенное дифференциальное поле.
\emph{Кольцом дифференциальных многочленов} от одной неизвестной над $K$
называется кольцо многочленов от счетного числа переменных ${K}[y_0,\ldots,y_n,\ldots]$
(обозначается ${K}\{y\}$)
с дифференцированием $\,\delta$, таким, что $\,\delta(y_i)=y_{i+1}$ и $\,\delta$, ограниченное на ${K}$, совпадает с дифференцированием поля $K$. Дифференцирование $\,\delta$, заданное таким способом, однозначно продолжается на $K\{y\}$ по линейности и правилу Лейбница.
\end{definition}

В дальнейшем для удобства мы будем обозначать $n$-ю производную многочлена $f$ как $\,\delta^n f$.

\begin{definition}
Пусть $M = \prod\limits_{i=0}^ny_i^{a_i}$, где $a_i \ge 0$, причем $a_n > 0$. Тогда $n$ называется \emph{порядком} монома $M$ (обозначение: $\ord M=n$). Соответственно, порядком многочлена называется максимальный порядок
мономов, его составляющих.
\end{definition}

Будем говорить, что переменная $y_i$ старше переменной $y_j$ (обозначение: $y_i \succ y_j$), если $i>j$.

\begin{definition}
\emph{Дифференциальным идеалом} дифференциального кольца $\mathcal{R}$ называется идеал $I$, такой, что $\,\delta(I)\subset I$.
\end{definition}

Будем обозначать через $(F)$ и $[F]$ сответственно алгебраический и дифференциальный идеалы, порождённые элементами из $F$.

\begin{definition}
Для элемента $f$ кольца $\mathbb{K}\{y\}$ определим \emph{сепаранту} $S_f$
как частную производную по старшей переменной. Для элементов поля $K$ полагаем ее нулем.
\end{definition}

\begin{definition} Многочлен называется \emph{квазилинейным}, если
его сепаранта~--- ненулевой элемент поля $K$.
\end{definition}



\section{Обобщенные сепаранты}
\label{generalized_separants}
Далее мы будем использовать букву $n$ для обозначения порядка дифференцирования и $l$ для порядка многочлена.

Известно, что производные дифференциального многочлена <<линейны>> по своей старшей переменной. Более того, коэффициент перед этой переменной для любого порядка дифференцирования один и тот же~--- это сепаранта:
\begin{gather}
\label{separant_formula}
 \,\delta^n f = S_f y_{l+n} + Q_{f,n},
\end{gather}
где $\ord Q_{f,n} < l + n$.
Мы обобщим это наблюдение на переменные, <<предшествующие>> старшей.


\begin{lemma}\label{lemma:der_monom}
Пусть моном $M$ записан в виде $M=y_{r_1}\ldots y_{r_d}$. Тогда
$$
\,\delta^n M =\sum\limits_{\stackrel{s_i\ge0,}{
s_1+\ldots+s_d=n}}\frac{n!}{s_1!\ldots s_d!}
\prod\limits_{i=1}^dy_{r_i+s_i}.
$$
\end{lemma}
\begin{proof}
Следует непосредственно из правила Лейбница.
\end{proof}


\begin{lemma}\label{lemma:commutative derivatives}
Пусть $f$~--- дифференциальный многочлен. Тогда для любого $s\in\mathbb{N}$ 
$$
\frac{\partial \,\delta f}{\partial y_s}=\,\delta \left(\frac{\partial f}{\partial y_s}\right)+\frac{\partial f}{\partial y_{s-1}},
$$
где $\frac{\partial}{\partial y_s}$~--- обычная частная производная.
\end{lemma}
\begin{proof}
Пусть $l=\ord f$.
Воспользуемся формулой $\,\delta f=\sum\limits_{i=0}^l\frac{\partial f}{\partial y_i}y_{i+1}$, которая следует из правила Лейбница. Имеем 
$$
\frac{\partial \,\delta f}{\partial y_s}=\frac{\partial\left(\sum\limits_{i=0}^l\frac{\partial f}{\partial y_i}y_{i+1}\right)}{\partial y_s}=\sum\limits_{i=0}^l\frac{\partial \frac{\partial f}{\partial y_i} }{\partial y_s}y_{i+1}+\sum\limits_{i=0}^l\frac{\partial y_{i+1}}{\partial y_s}\frac{\partial f}{\partial y_i}.
$$
Во второй сумме только слагаемое 
$\frac{\partial y_{s}}{\partial y_s}\frac{\partial f}{\partial y_{s-1}}=\frac{\partial f}{\partial y_{s-1}}$ 
не равно 0, т.~е. 
$$
\frac{\partial \,\delta f}{\partial y_s}=\sum\limits_{i=0}^l\frac{\partial \frac{\partial f}{\partial y_i}}{\partial y_s}y_{i+1}+\frac{\partial f}{\partial y_{s-1}}=\sum\limits_{i=0}^l\frac{\partial \frac{\partial f}{\partial y_s}}{\partial y_i}y_{i+1}+\frac{\partial f}{\partial y_{s-1}}=\,\delta\left(\frac{\partial f}{\partial y_s}\right)+\frac{\partial f}{\partial y_{s-1}}.
$$
\end{proof}

\begin{lemma}\label{lemma:f^(n)}
Пусть $f$~--- дифференциальный многочлен. Тогда для любых $n, s \in\mathbb{N}$ верно
$$
\frac{\partial \,\delta^n f}{\partial y_s} = \sum\limits_{i=0}^n  C_n^i  \,\delta^i \left(\frac{\partial f}{\partial y_{i+s-n}}\right).
$$
\end{lemma}
\begin{proof}
Докажем это утверждение индукцией по $n$.  База ($n=1$) верна по лемме~\ref{lemma:commutative derivatives}. Предположим, что для $n=k$ утверждение верно.
Докажем, что утверждение верно при $n=k+1$. Пусть $\,\delta f=g$. Тогда
$\frac{\partial \,\delta^{k+1} f}{\partial y_s} = \frac{\partial \,\delta^k g}{\partial y_s}$.
Применим предположение индукции к многочлену $\,\delta^k g$. Имеем 
$$
\frac{\partial \,\delta^{k+1} f}{\partial y_s} =  \sum\limits_{i=0}^k  C_k^i  \,\delta^i \left(\frac{\partial g}{\partial y_{i+s-k}}\right)=\sum\limits_{i=0}^k  C_k^i  \,\delta^i \left(\frac{\partial \,\delta f}{\partial y_{i+s-k}}\right). 
$$
Воспользуемся линейностью дифференцировния и леммой~\ref{lemma:commutative derivatives} в каждом слагаемом в последней сумме:  
$$
\,\delta^i \left(\frac{\partial \,\delta f}{\partial y_{i+s-k}}\right)=\,\delta^i\left(\frac{\partial f}{\partial y_{i+s-k-1}}+\,\delta\left(\frac{\partial f}{\partial y_{i+s-k}}\right)\right)=\,\delta^i\left(\frac{\partial f}{\partial y_{i+s-k-1}}\right)+\,\delta^{i+1}\left(\frac{\partial f}{\partial y_{i+s-k}}\right).
$$ 
Следовательно,
\begin{gather*}
\frac{\partial \,\delta^{k+1} f}{\partial y_s}=\sum\limits_{i=0}^k  C_k^i \,\delta^i \left(\frac{\partial f}{\partial y_{(i+s-k)-1}}\right)+\sum\limits_{i=0}^k  C_k^i \,\delta^{i+1} \left(\frac{\partial f}{\partial y_{i+s-k}}\right) = \\
= \sum\limits_{i=0}^k  C_k^i \,\delta^i \left(\frac{\partial f}{\partial y_{(i+s-k)-1}}\right)+\sum\limits_{i=1}^{k+1}  C_k^{i-1}\,\delta^i \left(\frac{\partial f}{\partial y_{i+s-k-1}}\right)=\sum\limits_{i=1}^{k+1} C_{k+1}^{i}\,\delta^i\left(\frac{\partial f}{\partial y_{i+s-(k+1)}}\right).
\end{gather*}
Переход доказан.
\end{proof}
Рассмотрим дифференциальный многочлен $g$ порядка $l$ и произвольное
целое число $k\ge0$. Представим многочлен в виде $g=g_0+A_{g,k}y_k$, где
$g_0$ не зависит от $y_k$, а многочлен $y_kA_{g,k}$ получается из
многочлена $g$ нахождением всех мономов, зависящих от $y_k$, и
вынесением этой переменной за скобки.
\begin{lemma}\label{lemma:A_fn}
Пусть $f$~--- дифференциальный многочлен и $\ord f = l$.
Тогда для любых $k$ и $n$ с условием  $n\in\mathbb{N},0\le k\in\mathbb{Z}$, $n>2k$ верно следующее
равенство:
$$
A_{\,\delta^n f, n+l-k}=\sum\limits_{j=0}^{l}C_n^{k - l + j
}\,\delta^{k-l+j}\left(\frac{\partial f}{\partial y_j}\right).
$$
Здесь мы по умолчанию понимаем $C_n^{k}=0$ для отрицательных
$k$.
\end{lemma}


\begin{proof}
Заметим, что в силу линейности дифференцирования мы можем считать
$f$ мономом. Пусть $\deg f = d$ и $f$ представляется в
виде
$$
f=\prod\limits_{i=0}^ly_i^{m_i}=\prod\limits_{i=1}^{d}y_{r_i},
$$
где $r_i\le r_j$ при $i \le j$. Тогда применив лемму~\ref{lemma:der_monom} к $\,\delta^n f$,
получим
$$
\,\delta^n f=\sum\limits_{\stackrel{s_i\ge0,}{
s_1+\ldots+s_d=n}}\frac{n!}{s_1!\ldots s_d!}
\prod\limits_{i=1}^dy_{r_i+s_i}.
$$
Так как $n>2k$, то в каждом мономе вида $\prod_{i=1}^d y_{r_i+s_i}$
переменная $y_{n+l-k}$ встречается единожды. Действительно, пусть
$y_{r_i+s_i}=y_{n+l-k}$ и $y_{r_j+s_j}=y_{n+l-k}$. Но тогда
$$
r_i+s_i+r_j+s_j = 2(n+l-k).
$$
Следовательно,
$$
2(n+l-k)\le 2r_d + \sum_{i=1}^d s_d = 2l +n.
$$
Значит, $n \le 2k$, что противоречит выбору $n>2k$. Мы получили, что многочлен $\,\delta^n f$ линеен по всем переменным $y_{n+l-k},n>2k $. Пусть $\,\delta^n f=f_0+A_{\,\delta^n f,n+l-k}y_{n+l-k}$. Тогда, в силу линейности $\,\delta^n f$ по $y_{n+l-k}$,
$$
\frac{\partial \,\delta^n f}{\partial y_{n+l-k}}=\frac{\partial (f_0+A_{\,\delta^n f,n+l-k}y_{n+l-k})}{\partial y_{n+l-k}}=A_{\,\delta^n f,n+l-k}.
$$

С другой стороны, по лемме~\ref{lemma:f^(n)} $\frac{\partial \,\delta^n f}{\partial y_{n+l-k}}= \sum\limits_{i=0}^n  C_n^i  \,\delta^i\left(\frac{\partial f}{\partial y_{i+l-k}}\right)$. Заметим, что так как $\ord f = l$, то для всех $i>k$ выполнено $\frac{\partial f}{\partial y_{i+l-k}}=0$. Отсюда следует, что 
\begin{gather*}
A_{\,\delta^n f,n+l-k}=\sum\limits_{i=0}^n  C_n^i  \,\delta^i\left(\frac{\partial f}{\partial y_{i+l-k}}\right)=\sum\limits_{i=0}^k  C_n^i  \,\delta^i \left(\frac{\partial f}{\partial y_{i+l-k}}\right)=\\
=\sum\limits_{j=l-k}^l  C_n^{k-l+j}  \,\delta^{k-l+j}\left(\frac{\partial f}{\partial y_{j}}\right)=\sum\limits_{j=0}^{l}C_n^{k - l + j
}\,\delta^{k-l+j}\left(\frac{\partial f}{\partial y_j}\right).
\end{gather*}
Последнее равенство верно, так как если $l>k$, то из $0 \le j<l-k \Longrightarrow C_n^{k-l+j}=0$, а если $l<k$, то  $k-l\le j<0 \Longrightarrow \frac{\partial f}{\partial y_{j}}=0$. 
\end{proof}

Пусть $f$~--- дифференциальный многочлен, $\ord f = l$, $n\in\mathbb{N}$, $0\le k\in\mathbb{Z}$ и $n>2k$.
Введём более удобное обозначение
\begin{gather*}
 S_{f,n,k} := A_{\,\delta^n f, l+n-k} = \sum_{j=\max(l-k,0)}^l C_{n}^{k-l+j} \,\delta^{k-l+j}\left(\frac{\partial f}{\partial y_{j}}\right)
\end{gather*}
(по лемме~\ref{lemma:A_fn}).
Коэффициенты $S_{f,n,k}$ будем называть \emph{обобщёнными сепарантами} многочлена $f$.
В частности, при $k=0$ обобщённая сепаранта совпадает с обычной: $S_{f,n,0}=S_f = \frac{\partial f}{\partial y_l}$. 
Заметим, что при указанных предположениях $\ord S_{f,n,k} \le k+l$.

\begin{proposition}
\label{generalized_separants_expansion}
При $n > 2p$
\begin{gather}
\label{generalized_separant_formula}
 \,\delta^n f = S_{f,n,0} \, y_{l+n} + S_{f,n,1} \, y_{l+n-1} + \ldots + S_{f,n,p} \, y_{l+n-p} + T_{f,n,p},
\end{gather}
где $\ord T_{f,n,p} < l+n-p$.
\end{proposition}

\begin{proof}
Докажем утверждение индукцией по $p$.
База выполнена: при $p = 0$ утверждение превращается в формулу~\eqref{separant_formula}.
Предположим, что для некоторого $p$ утверждение доказано, и докажем его для $p + 1$, где $2(p+1) < n$.
Заметим, что 
$$
 \ord S_{f,n,k} \le k + l \le p + l < l+n-p-1,
$$
поэтому переменная $y_{l+n-p-1}$ не встречается в слагаемых вида $S_{f,n,k} \, y_{l+n-k}$ формулы~\eqref{generalized_separant_formula}.
Она может встретиться только в $T_{f,n,p}$, причём в этом случае она окажется старшей переменной $T_{f,n,p}$.
Значит,
$$
 S_{f,n,p+1} = A_{\delta^n f, \, l+n-p-1} = A_{T_{f,n,p}, \, l+n-p-1},
$$
причем 
$$
 T_{f,n,p} = S_{f,n,p+1} \, y_{l+n-p-1} + T_{f,n,p+1},
$$
где $\ord T_{f,n,p+1} < l+n-p-1$.
\end{proof}

\bigskip


\section{Алгоритм, проверяющий равенство~\eqref{strong_general}}
\label{algorithm}

Сформулируем сначала несколько утверждений об идеалах в произвольном коммутативном кольце.

\begin{proposition}
Если $S$~--- идеал, а $I$~--- радикальный идеал, то $I = I:S \cap (I+S)$. $\qed$
\end{proposition}

\begin{proposition}
Если $I$~--- идеал, то $I : (g_1, \ldots, g_n) = \bigcap_{i=1}^n I:(g_i)$. $\qed$
\end{proposition}

\begin{proposition}
\label{Primes}
Если $Q_i$~--- идеалы, $P$~--- простой идеал и $\bigcap_{i=1}^n Q_i \subset P$, то для некоторого $j$ $Q_j \subset P$. $\qed$
\end{proposition}


\bigskip

Пусть даны идеалы в кольце многочленов $K[y_0, \ldots, y_s]$.
Очевидно, что такие свойства идеалов, как тривиальность, простота, радикальность, включение
не зависят от присоединения к кольцу дополнительных независимых переменных.
Поэтому, говоря об идеалах, мы не будем уточнять, в кольцах многочленов от каких переменных они рассматриваются.


\bigskip

Перейдём теперь к формулировке задачи.

Пусть $f \in \mathbb{Q}\{y\}$~--- дифференциальный многочлен порядка $l$.
Пусть $h_1, \ldots, h_t$~--- произвольные многочлены, образующие идеал $J$.
Пусть $G_k$~--- идеал, порожденный элементами $\,\delta^{k-l+j} \left(\frac{\partial f}{\partial y_j}\right)$, где $\max(l-k, 0) \le j \le l$.
Пусть также $S_p = G_0 + \ldots + G_p$.
Предположим, что $[f] + S_p = (1)$.
Требуется выяснить, равен ли идеал $I = [f] + J$ единичному.
Заметим, что разделение случаев~\eqref{strong} и~\eqref{nontrivial} является частным случаем этой задачи, когда $J = (S_f)$.

\bigskip


Пусть $I_k = (f_0, \ldots, f_k) + J$. Заметим, что 
$$
 I = (1) \iff \exists \, k: I_k = (1) \iff \exists \, k: \sqrt{I_k} = (1).
$$
Выберем теперь $k$ таким, чтобы $(f_0, \ldots, f_k) + S_p = (1)$, и чтобы $k$ превосходило порядки всех многочленов $h_i$
и было не меньше $2p$.
Проверим, выполняется ли равенство $I_k = (1)$.

Разложим идеал $\sqrt{I_k}$:
$$
 \sqrt{I_k} = \sqrt{I_k} : S_p  \cap  \left(\sqrt{I_k} + S_p\right).
$$
Так как $\sqrt{I_k} + S_p = (1)$, то
$$
 \sqrt{I_k} = \sqrt{I_k} : S_p = \sqrt{I_k} : (G_0 + \ldots + G_d) = \bigcap\limits_{d = 0}^p \sqrt{I_k} : G_d.
$$
Чтобы доказать, что этот идеал отличен от $(1)$, необходимо и достаточно доказать, что один из идеалов $\sqrt{I_k}:G_d$ отличен от $(1)$.
Будем перебирать идеалы $\sqrt{I_k} : G_d$, увеличивая $d$.

Если $\sqrt{I_k} : G_d = (1)$, то и $\sqrt{I_n} : G_d = (1)$ для всех остальных $n \ge k$.
Будем считать, что $d$~--- наименьший номер, такой, что $\sqrt{I_k} : G_d \ne (1)$.
Наша цель~--- постараться получить аналогичное неравенство для $k+1$.

Идеал $\sqrt{I_k} : G_d$ является радикальным. Разложим его в минимальное пересечение простых идеалов $P_{k,d,i}$.
Это в точности минимальные простые компоненты идеала $\sqrt{I_k}$, не содержащие $G_d$.
Предположим, что среди этих простых идеалов существует идеал $P$, не содержащий ни одной обобщенной сепаранты $S_{f,n,d}$.
Покажем, что тогда идеал $\sqrt{I_{k+1}}:G_d$ также является собственным, и среди его минимальных простых также есть идеал $Q$,
не содержащей обобщенных сепарант. В таком случае по индукции мы получим, что $I \ne 1$.

Действительно, так как $k+1 > 2p$, то к $f_{k+1}$ применимо предложение~\ref{generalized_separants_expansion}. 
Пусть $z = y_{l+k+1-d}$~--- старшая переменная многочлена $f_{k+1}$ по модулю $G_0, \ldots, G_{d-1}$
и $R$~--- подкольцо кольца многочленов, порожденное переменными, младшими $z$. 
Образующие идеалов $I_k, G_d$ и $P$ лежат в $R$.
Тогда $f_{k+1} \equiv S_{f,k+1,d} z + T \pmod P$, где $T \in R$.
Если $h \in I_{k+1} \cap R$, то верно равенство
$h \equiv q f_{k+1} \pmod P$, причем $q S_{f,k+1,d} \in P$.
Так как по предположению $S_{f,k+1,d} \notin P$ и $P$ является простым, то $q \in P$,
откуда $h \in P$. Итак,
\begin{gather*}
 I_{k+1} \cap R \subset P \; \Longrightarrow \; \sqrt{I_{k+1}} \cap R \subset \sqrt{P} = P \; \Longrightarrow \\
 \Longrightarrow \; \sqrt{I_{k+1}}:G_d \cap R \subset P:G_d = P \; \Longrightarrow \; \sqrt{I_k}:G_d \subset \sqrt{I_{k+1}}:G_d \cap R \subset P.
\end{gather*}
Пусть $\sqrt{I_{k+1}}:G_d = \cap_j Q_j$~--- минимальное разложение на простые компоненты.
Тогда все идеалы $Q_j \cap R$ тоже простые, причем
$$
 \sqrt{I_k}:G_d \subset \bigcap_j \left( Q_j \cap R \right) \subset P.
$$
Отсюда следует, что найдется $Q$, такой, что $Q \cap R = P$, так как $P$~--- минимальный простой.
Так как все обобщенные сепаранты лежат в $R$ и не лежат в $P$, то они не лежат и в $Q$.

\bigskip

Предположим теперь, что для каждого минимального простого идеала $P_{k,d,i}$
найдется принадлежащая ему обобщенная сепаранта $S_{f,n,d}$, где $n \ge k$.
Заметим, что все обобщенные сепаранты лежат в идеале $G_d$.
Выражение для этих обобщенных сепарант является многочленом по $n$ степени $d$,
не равным тождественно нулю по модулю каждого идеала $P_{k,d,i}$ (иначе получилось бы противоречие $G_d \subset P_{k,d,i}$).
Отсюда следует, что лишь для конечного набора значений $n \ge k$ обобщенная сепаранта $S_{f,n,d}$ может оказаться в идеале $P_{k,d,i}$.
Пусть число $K$ больше всех этих значений. Перейдем от $k$ к $K$ и повторим процедуру. Покажем, что такой повтор может произойти не более $d+1$ раз. Действительно, пусть имеется цепочка наборов минимальных простых идеалов $P_{k_j, d, i}$ для идеалов $\sqrt{I_{k_j}}:G_d$ длины $d+1$
$$
 \bigcap_i P_{k_1, d, i} \subset \bigcap_i P_{k_2, d, i} \subset \ldots \subset \bigcap_i P_{k_{d+1}, d, i},
$$
причем в каждом из них содержится некоторая обобщенная сепаранта.
Выберем по предложению~\ref{Primes} цепочку вложенных простых идеалов:
$$
 P_{k_1, d, i_1} \subset P_{k_2, d, i_2} \subset \ldots \subset P_{k_{d+1}, d, i_{d+1}}.
$$
Рассмотрим выражение для обобщенной сепаранты по модулю последнего идеала в этой цепочке. Это не тождественный ноль,
а некоторый многочлен по $n$ степени не выше $d$, имеющий корень по крайней мере в $d+1$ точке, что немедленно приводит нас к противоречию.
Итак, по крайней мере за $d$ таких повторов мы либо убедимся, что $\sqrt{I_k}:G_d = (1)$, либо получим простую компоненту этого идеала, не содержащую никаких обобщенных сепарант, и тогда по доказанному выше $I \ne (1)$.

\bigskip

Запишем проведённые рассуждения в виде алгоритма.

%\algrenewcommand\algorithmicwhile{\textbf{am\’\i g}}
\begin{algorithm}
\noindent {\bf Вход: }{многочлен $f \in K\{y\}$ порядка $l$, такой, что $\left[f,\frac{\partial f}{\partial y_0},\ldots, \frac{\partial f}{\partial y_l}\right]=(1)$;\\
\phantom{{\bf Вход:}} многочлены $h_1, \ldots, h_t \in K\{y\}$}

\noindent {\bf Выход:} \verb'True', если  $[f]+(h_1,\ldots,h_t)=(1)$, и \verb'False', если $[f]+(h_1,\ldots,h_t)\ne(1)$.
\begin{algorithmic}[1]
\Statex
\State\label{find} {\bf find} $k$ {\bf and} $p$ {\bf such that} $(f, \ldots, \,\delta^k f) + G_0 + \ldots + G_p = (1)$, $k \ge 2p$, {\bf and} $k > \ord h_i$ {\bf for all} $i$
\State $I_k \gets (f, \ldots, \,\delta^k f, h_1, \ldots, h_t)$
\For{$d \gets 0, \ldots, p$}
  \If{$G_d \subset \sqrt{I_k}$}\label{subset}
    \State {\bf continue}
  \EndIf
  \State\label{primes} $primes \gets$ \Call{MinimalPrimeDecomposition}{$\sqrt{I_k}:G_d$}
  \State $S_{f,n,d}(n) \gets \sum_{j=\max(l-d,0)}^l C_{n}^{d-l+j} \,\delta^{d-l+j}\left(\frac{\partial f}{\partial y_{j}}\right)$
  \State $K \gets k$
  \ForAll{$P \in primes$}
    \State\label{remainder} $r(n) \gets$ \Call{Reduce}{$S_{f,n,d}(n), P$}
    \State $n_{\max} \gets$ \Call{MaxIntegerRoot}{$r(n)$}
    \If{$n_{\max} < k$}
      \State\label{return_false} \Return{\verb'False'}
    \EndIf
    \State $K \gets \max(K, n_{\max})$
  \EndFor
  \State $k \gets K + 1$
  \State $I_k \gets (f, \ldots, \,\delta^k f, h_1, \ldots, h_t)$
\EndFor
\State \Return{\verb'True'}
\end{algorithmic}
\end{algorithm}


\bigskip

\section{Примеры}
В приводимых ниже выкладках можно убедиться с помощью любой системы компьютерной алгебры, позволяющей вычислять базисы Грёбнера.

\label{examples}
\begin{example}\label{example:first}
Рассмотрим дифференциальный многочлен 
$$
f= y_0 y_1^2+ay_1^2+by_0 y_1+c,
$$
где $c \ne 0$.
Выясним, при каких константных значениях параметров $a$, $b$ и $c$ выполнено $[f]+(S_f)=(1)$.
Хотя наш алгоритм не предполагает наличия параметров, мы решим эту задачу в общем виде.

Можно проверить, что $[f] + [S_f] = (1)$.
В самом деле, $c^2$ полиномиально выражается через систему $f, \delta f, \delta^2 f, S_f, \delta {S_f}$.
Поэтому на шаге~\ref{find} можно выбрать $p = 1$ и $k=2$.
Соответствующие идеалы $G_d$ и $I_k$ выглядят так:
\begin{gather*}
G_0 = (S_f),\\
G_1 = \left({S_f}', \frac{\partial f}{y_0} \right),\\
I_2 = \left(f, \delta f, \delta^2 f, S_f\right).
\end{gather*}
Они рассматриваются в кольце $\mathbb{Q}[y_0, y_1, y_2, y_3]$ (при фиксированных значениях параметров).
Заметим, что $G_0 \subset I_2$, поэтому на шаге~\ref{subset} для $d=0$ включение будет выполнено и алгоритм перейдет к $d=1$.
Построим идеал $\sqrt{I_2}:G_1$.
Рассматривая его в $\mathbb{Q}[a,b,c,y_0,y_1, y_2, y_3]$ и исключая переменные $y_j$, получим многочлен $c + a b^2$.
Итак, если $c + ab^2 \ne 0$, то $\sqrt{I_2}:G_1 = (1)$, то есть, включение на шаге~\ref{subset} снова выполнено,
цикл закончится и алгоритм вернет \verb'True'. В этом случае $[f] + (S_f) = (1)$.

Пусть теперь $c + a b^2 = 0$.
В этом случае алгоритм должен на шаге~\ref{primes} построить разложение идеала $\sqrt{I_2}:G_1$
на минимальные простые компоненты. 
Выполним эту операцию в кольце $\mathbb{Q}[a,b,c,y_0,y_1,y_2,y_3]$.
В результате мы получим разложение идеала $\sqrt{I_2}:G_1$ на две компоненты,
которые, однако, могут утратить свойство простоты при подстановке вместо параметров $a, b$ и $c$ конкретных значений.
Тем не менее, уже в этом случае остатки $r(n)$ для каждой из компонент на шаге~\ref{remainder} дают многочлены,
пропорциональные $b^2 n$, причем по предположению $b^2 \ne 0$, так как $c \ne 0$.
Единственный целый корень $0$ этих многочленов меньше текущего значения $k = 2$,
поэтому алгоритм на шаге~\ref{return_false} вернет \verb'False', то есть,
при $c + a b^2 = 0$ идеал $[f] + (S_f)$ отличен от всего кольца.
\end{example}


\begin{example}
Рассмотрим многочлен $f = (y_1-a)^2 - b^2 y_0^4 y_1^2$,
где коэффициенты $a$ и $b$ отличны от нуля.
Заметим, что $ab$
полиномиально выражается через $f, \ldots, \,\delta^4 f, S_f, \,\delta {S_f}, \,\delta^2 {S_f}$.
Значит, при любых ненулевых $a$ и $b$ идеал $[f] + [S_f]$ тривиален.
Здесь нам потребовалось дважды продифференцировать сепаранту $S_f$.
Выясним, возможна ли ситуация $[f] + (S_f, \delta {S_f}) = (1)$.

На шаге~\ref{find} можно выбрать $p = 2$ и $k=4$.
Тогда
$$
 I_4 = (f, \ldots, \,\delta^4 f, S_f, \,\delta {S_f}).
$$
Заметим, что при $d=0$ и $d=1$ идеалы $\sqrt{I_4}:G_d$ оказываются тривиальными.
Переходим к последней итерации внешнего цикла алгоритма при $d=2$.
В кольце $\mathbb{Q}[a,b,y_0,y_1,y_2,y_3]$
идеал $\sqrt{I_4}:G_2$ раскладывается в пересечение двух простых компонент, которые порождаются элементами
$$
 y_0, \, y_1 - a, \, y_2, \, y_3 - 2 a^3 b
$$
и
$$
 y_0, \, y_1 - a, \, y_2, \, y_3 + 2 a^3 b.
$$
Видно, что при конкретных значениях параметров $a$ и $b$
эти идеалы остаются простыми в кольце $\mathbb{Q}[y_0, y_1, y_2, y_3]$.
Остаток от редукции многочлена $S_{f,n,2}(n)$
относительно первого из них равен $2a^3 b n(n-1)$.
Целые корни этого многочлена, $0$ и $1$, меньше текущего значения $k=4$,
поэтому алгоритм вернет \verb'False' в строке~\ref{return_false}.
Итак, при всех ненулевых $a$ и $b$ идеал $[f] + (S_f, \delta {S_f})$ отличен от единичного.
\end{example}

\begin{thebibliography}{99}

\bibitem{AM}
М. Атья, И. Макдональд.
\emph{Введение в коммутативную алгебру.} М., <<Мир>>, 1972.

\bibitem{Cox}
Д. Кокс, Д. Литтл, Д. О'Ши.
\emph{Идеалы, многообразия и алгоритмы}.
М., <<Мир>>, 2000.

\bibitem{Trushin}
Д. В. Трушин.
\emph{Идеал сепарант в кольце дифференциальных многочленов}.
Фундаментальная и прикладная математика, том~13, вып.~1, 215--227, 2007.

\bibitem{BLOP}
F. Boulier, D. Lazard, F. Ollivier, M. Petitot.
\emph{Representation for the Radical of a Finitely Generated Differential Ideal}.
In Proceedings of 1995 International Symposium on Symbolic and Algebraic Computation (ISSAC-1995), ACM Press, 158--166, 1995.
 
\bibitem{CarraFerro}
G. Carr\`a Ferro.
\emph{Diferential Gr\"{o}bner Bases in One Variable and in the Partial Case}.
Math. Comput. Modelling, Pergamon Press, vol.~25, 1--10, 1997.

\bibitem{Kolchin41}
E. R. Kolchin.
\emph{On the Exponents of Differential Ideals}.
The Annals of Mathematics, Second Series, vol.~42 (3), 740--777, 1941.

\bibitem{Kolchin73}
E. R. Kolchin.
\emph{Differential algebra and algebraic groups}.
Academic Press, 1973.

\bibitem{Sit}
W. Sit.
\emph{The Ritt-Kolchin Theory for Differential Polynomials}.
Differential Algebra and Related Topics, 2000.

\bibitem{Zobnin} 
A. Zobnin. 
\emph{Admissible Orderings and Finiteness Criteria for Differential Standard Bases}. 
In Proceedings of International Symposium on Symbolic and Algebraic Computation (ISSAC-2005), ACM Press,
365--372 (2005).
\end{thebibliography}
\end{document}
\begin{definition}
\end{definition}
